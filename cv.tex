
\documentclass[a4paper, 12pt, color, final]{moderncv}

\moderncvstyle{classic}
\definecolor{color0}{rgb}{0,0,0}
\definecolor{color1}{rgb}{0.50,0.44,0.60}
\definecolor{color2}{rgb}{0.55,0.55,0.55}

\nopagenumbers{}

\usepackage[utf8]{inputenc}

\usepackage[scale=0.80]{geometry}
\setlength{\hintscolumnwidth}{3cm}

% personal data
\name{Andrei}{Tsialiatka}
\title{Software engineer}
\address{Minsk}{Belarus}
\email{steepashka1love@gmail.com}
\social[linkedin]{atelyatko}
\social[github]{letmecnsooit}
\photo[90pt][0pt]{me.png}

\begin{document}
\makecvtitle

\section{Employment History}
\cventry{November 2022 -- Present\hfill}{Senior Software Engineer}{ControlUp}{Minsk}{}{}
\cventry{September 2017 -- September 2022\hfill}{Software Developer}{Itransition Group}{Minsk}{}{}

\section{Projects}

\cventry{2022--2024}{ControlUp}{}{}{}{
\begin{itemize}%
\item ControlUp is a comprehensive IT management and monitoring solution designed to enhance the performance and stability of virtual environments. It provides real-time monitoring, troubleshooting, and analytics for virtual desktop infrastructure (VDI) and server virtualization environments, such as Citrix XenDesktop/XenApp, VMware Horizon, and Microsoft RDS. ControlUp offers a range of features, including real-time monitoring of virtual and physical resources, user experience metrics, application performance insights, and detailed historical data analysis. Additionally, ControlUp provides automation and management features, enabling IT professionals to automate routine tasks, set up alerts for specific events, and conduct various management operations across multiple environments from a central console.
\item \textbf{Tools}: .NET 8, PostgreSQL, MS SQL, Redis, EF Core, log4net, Splunk, GrayLog, .NET 4.8, WinForms, WPF, Docker, gRPC, React.
\end{itemize}}

\cventry{2021--2022}{Property renting}{}{}{}{
\begin{itemize}%
\item The application is introducing greater choice for renters based on their individual preferences.
It provides a supportive solution for those seeking a replacement to a traditional cash deposit that can adapt to tenants individual circumstances and maintain their cash flow at the start, and crucially at the end of the tenancy.
The application offers tenancy deposit choice via its Deposit Replacement Membership – a monthly subscription-style service for tenants which allows them to rent a home without a physical deposit whilst also protecting a landlord’s need for financial security.
One of the main advantages of the application is that it provides API-interfaces for automating the journey for landlords and tenants – with a handful of early candidates already lined up.
\item \textbf{Tools}: .NET 6, Azure SQL, EF Core, Autofac, AutoMapper, MediatR, xUnit, Identity Server, Serilog, Azure Storage, Azure MySQL, Azure AppService, Azure Service Bus, Azure Key Vault, Strapi CMS, TypeScript, React, Material UI, Storybook, Mobx, Yup, React Hook Form, Jest, Fluent Validator, SignalR, Azure Media Services.
\end{itemize}}

\cventry{2020--2021}{Chromatography results processing}{}{}{}{
\begin{itemize}%
    \item A next-gen version of the original application which allows users to process and
    control different steps and stages of chromatography results. The application
    provides an ability to create plots for chromatography result peaks, apply integration
    filters for peaks, do system suitability checks to verify the results of integration and
    calibration steps, create user-specific workflows, apply permissions for users to give
    an access to work on a specific area of processing.
    \item \textbf{Tools}: ASP.NET Core 3.1, AWS (DocumentDB, S3), Docker, Angular 11, Typescript,
    D3.js, Angular Material, Wijmo, Karma, Jasmine, SpecFlow.
\end{itemize}}

\cventry{2017--2020}{Healthcare assistant}{}{}{}{
\begin{itemize}
    \item A complex healthcare solution that is aimed at helping people with chronic
    pain. Development includes several mobile applications, as well as separate web
    application along with a range of custom RESTful APIs. Main features of the project:
    \begin{itemize}
        \item RTC communication (audio/video calling, text chatting across multiple platforms);
        \item Activity tracker integration (Fitbit, Apple Health);
        \item Assessment module that defines a custom course of actions for the patient;
        \item Elaborate reporting/statistics modules for patients and counselors;
        \item HIPAA compliance;
    \end{itemize}
    \item \textbf{Tools}: ASP.NET Web API 2, Angular 6, Bootstrap 3, Dapper ORM, TokBox, Kendo
    UI, Telerik Reporting Tools, OData v4, C\# 7, TypeScript, Karma, Jasmine, Git.
\end{itemize}}

\cventry{2017--2020}{Utilization review processing}{}{}{}{
\begin{itemize}
    \item A service that provides prospective, concurrent and retrospective review of the
    medical necessity of services. The project facilitates the timely and appropriate
    treatment of injured employees while providing employers with greater financial
    controls; Provides an ability to estimate the cost of physicians and nurses working
    processes based on different UR cases criterias; Allows to generate a huge amount
    of reports with different metrics and criterias. Provides an ability to determine state
    workers compensation plans by using industry standard guidelines such as ACOEM
    \& ODG. Permission-based engine to restrict users to have an access to different
    application’s areas.
    \item \textbf{Tools}: JetBrains Rider IDE, Git, Bitbucket, IIS 7, MS SQL Server, SQL Server Profiler,
    SQL Server Management Studio, Postman, ASP.NET Web API 2, Entity Framework
    6, OData v4, DocX, EPPlus, Telerik Reporting Tools, Autofac, NUnit, AngularJS 1.6,
    KendoUI, Twitter Bootstrap 3, LESS, Webpack, npm, KendoUI, Babel, HTML5, CSS3,
    Jasmine.
\end{itemize}}

\section{Education}
\cvitem{2015--2019}{
    \textbf{Belarussian State University of Informatics and Radioelectronics}; \newline
    Faculty of Computer Systems and Networks; \newline
    Software for Information Technologies; \newline
    Bachelor's degree.
}
\cvitem{2019--2020}{
    \textbf{Belarussian State University of Informatics and Radioelectronics}; \newline
    Faculty of Computer-Aided Design; \newline
    Labor Protection and Ergonomics; \newline
    Master's degree.
}

\section{Skills}
\subsection{Main:}
\cvitem{}{ASP.NET Core, Angular, Azure, SQL, Git, NoSQL, Docker, AWS, React, TypeScript. }
\subsection{Experience with:}
\cvitem{}{TeamCity, Azure DevOps, Go, Python, \LaTeX. }

\section{Languages}
\cvitemwithcomment{English}{Advanced}{}
\cvitemwithcomment{Belarusian}{Native speaker}{}
\cvitemwithcomment{Russian}{Native speaker}{}

\end{document}
